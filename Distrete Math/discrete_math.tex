\documentclass[11pt]{article}
\usepackage[margin=1in]{geometry}
\usepackage{amsmath, amssymb, amsthm, esint}
\usepackage{fancyhdr}
\usepackage{tikz, tikz-3dplot}
% \usepackage{hyperref}
\usepackage{enumitem}
\usepackage{float}
\usepackage{booktabs}
\usepackage{array}
\usepackage{cancel}

\setlength{\headheight}{14pt}
\fancyhf{}
\lhead{Discrete Mathematics}
\cfoot{\thepage}

\begin{document}
\pagestyle{plain}
\begin{center}
    \tableofcontents
\end{center}
\newpage
\setcounter{page}{1}
\pagestyle{fancy}
\section{Logic and Proofs}
\subsection{Propositional Logic}
Proposition is a statement that is \textbf{either} true or false, but not both at the same time. We usually represent it with variables like $p$, $q$, and $r$.\\
e.g. "The sky is blue." is a proposition, but "Listen to me" is not. 
\subsubsection{Logical Connectives}
\begin{itemize}
    \item Negation: $\lnot \,p$. It is not the case that $p$.
    \item Conjunction: $p \land q$. "and"
    \item Disjunction: $p \lor q$. "or"
    \item Implication: $p \rightarrow q$. If $p$ then $q$, $q$ if $p$, $q$ is a consequence of $p$, $p$ only if $q$
    \item biconditional: $p \leftrightarrow q$. $(p\rightarrow q)\land(q \rightarrow p)$, $p$ if and only if $q$
\end{itemize}
\subsubsection{Variations of Conditionals}
\begin{itemize}
    \item Implication: $p \rightarrow q$
    \item Converse: $q \rightarrow p$
    \item Inverse: $\lnot p \rightarrow \lnot q$
    \item Contrapositive: $\lnot q\rightarrow \lnot p$. This is logically equivalent to Implication
\end{itemize}
\subsubsection*{Truth Table}
\begin{table}[H]
    \centering
    \hfill
    \begin{minipage}{.32\textwidth}
        \centering
        \begin{tabular}{c|c|c}
            $p$ & $q$ & $p \lor q$\\
            \hline
            T & T & T\\
            T & F & T\\
            F & T & T\\
            F & F & F
        \end{tabular}
    \end{minipage}
    \hfill
    \begin{minipage}{.32\textwidth}
        \centering
        \begin{tabular}{c|c|c}
            $p$ & $q$ & $p \land q$\\
            \hline
            T & T & T\\
            T & F & F\\
            F & T & F\\
            F & F & F
        \end{tabular}
    \end{minipage}
    \hfill
    \begin{minipage}{.32\textwidth}
        \centering
        \begin{tabular}{c|c|c}
            $p$ & $q$ & $p \rightarrow q$\\
            \hline
            T & T & T\\
            T & F & F\\
            F & T & T\\
            F & F & T
        \end{tabular}
    \end{minipage}
    \hfill
\end{table}
\subsubsection*{Example}
Find the truth value of $\boldsymbol{(p \lor q)\rightarrow \lnot \,r}$
\begin{table}[H]
    \centering
    \begin{tabular}{c|c|c|c|c|c}
        $p$ & $q$ & $r$ & $p \lor q$ & $\lnot \,r$& $(p \lor q)\rightarrow \lnot\, r$ \\
        \hline
        T & T & T & T & F & F\\
        T & T & F & T & T & T\\
        T & F & T & T & F & F\\
        T & F & F & T & T & T\\
        F & T & T & T & F & F\\
        F & T & F & T & T & T\\
        F & F & T & F & F & T\\
        F & F & F & F & T & T
    \end{tabular}
\end{table}
\subsection{Application of Propositional Logic}
\subsubsection{Classification of Proposition}
\begin{itemize}
    \item Tautology: Always true. e.g. $p\lor \lnot\,p$
    \item Contradiction: Always false. e.g. $p\land \lnot\,p$
    \item Contingency: Depends on variable. e.g. $p\rightarrow q$
\end{itemize}
\subsubsection{Logical Equivalence $p \equiv q$}
Two statements are logically equivalent if they always have the same truth value in every possible scenario.\\
e.g. $p$ and $q$ are biconditional, i.e. $p \leftrightarrow q$, means that $p$ and $q$ are logically equivalent.
\subsubsection{Laws of Logical Equivalence}
\begin{table}[H]
    \centering
    \renewcommand{\arraystretch}{1.3}
    \begin{tabular}{>{$}l<{\quad$}|l}
        \text{Equivalence} & \text{Name} \\
        \hline
        p \land T \equiv p & Identity laws \\
        p \lor F \equiv p & \\
        \hline
        p \lor T \equiv T & Domination laws \\
        p \land F \equiv F & \\
        \hline
        p \lor p \equiv p & Idempotent laws \\
        p \land p \equiv p & \\
        \hline
        \lnot(\lnot p) \equiv p & Double negation law \\
        \hline
        p \lor q \equiv q \lor p & Commutative laws \\
        p \land q \equiv q \land p & \\
        \hline
        (p \lor q) \lor r \equiv p \lor (q \lor r) & Associative laws \\
        (p \land q) \land r \equiv p \land (q \land r) & \\
        \hline
        p \lor (q \land r) \equiv (p \lor q) \land (p \lor r) & Distributive laws \\
        p \land (q \lor r) \equiv (p \land q) \lor (p \land r) & \\
        \hline
        \lnot(p \land q) \equiv \lnot p \lor \lnot q & De Morgan's laws \\
        \lnot(p \lor q) \equiv \lnot p \land \lnot q & \\
        \hline
        p \lor (p \land q) \equiv p & Absorption laws \\
        p \land (p \lor q) \equiv p & \\
        \hline
        p \lor \lnot p \equiv T & Negation laws \\
        p \land \lnot p \equiv F & \\
        \hline
        p \rightarrow q \equiv \lnot p \lor q & Conditional\\
        p \rightarrow q \equiv \lnot q \rightarrow \lnot p & \\
        \hline
        p \leftrightarrow q \equiv (p \rightarrow q) \land (q \rightarrow p) & Biconditional\\
        p \leftrightarrow q \equiv \lnot p \leftrightarrow \lnot q & \\
    \end{tabular}
\end{table}
\subsubsection{Determine Logical Equivalence: }
\begin{enumerate}
    \item Verify with Truth Table
    \item Apply Known knowledge
\end{enumerate}
\subsubsection*{Show that $p\rightarrow q$ is logically equivalent to $\lnot q \rightarrow \lnot p$}
\begin{table}[H]
    \centering
    \begin{tabular}{c|c|c|c}
        $p$ & $q$ & $p \rightarrow q$ & $\lnot q\rightarrow \lnot p$ \\
        \hline
        T & T & T & T\\
        T & F & F & F\\
        F & T & T & T\\
        F & F & T & T\\
    \end{tabular}
\end{table}
\subsubsection*{Show that $(p\rightarrow r)\lor (q\rightarrow r) \equiv (p\land q)\rightarrow r$}
\begin{align*}
    (p\rightarrow r)\lor (q\rightarrow r) &\equiv(\lnot p \lor r)\lor (\lnot q \lor r)\\
    &\equiv(\lnot p \lor \lnot q)\lor(r\lor r)\\
    &\equiv\lnot(p\land q) \lor r\\
    &\equiv (p\land q)\rightarrow r _\#
\end{align*}
\subsection{Predicate and Quantifier}
\subsubsection{Predicate}
A predicate is a statement with variables that becomes true or false only once specific values are substituted. $P(x)$ denotes a predicate involving $x$.\\
e.g. Let $P(x)$ be the statement "x>4." We read $P(x)$ as "$x$ is greater than $4$."
\begin{itemize}
    \item $P(x)$ is true if x = 5
    \item $P(x)$ is false if x = 3
\end{itemize}
\subsubsection*{General Form}    
$P(x_1, x_2, x_3, \dots, x_n)$ where each $x_i$ is a variable from the domain of discourse. 
\subsubsection{Quantifier}
\begin{itemize}
    \item Universal quantifier $\forall$: "for all", "every".
    \item Existential quantifier $\exists$: "there exists", "some", "at least one".
\end{itemize}
\subsubsection*{Negating Quantifier}
\begin{align*}
    \begin{cases}
        \lnot \forall x \,P(x) \equiv \exists x \,\lnot P(x)\\
        \lnot \exists x \,P(x) \equiv \forall x\, \lnot P(x)
    \end{cases}
\end{align*}
\subsubsection*{Nested Quantifier}
\[
    \forall x \exists y \,P(x, y) \neq\exists y \forall x \,P(x)
\]

\end{document}