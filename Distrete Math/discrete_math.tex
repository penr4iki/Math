\documentclass[11pt]{article}
\usepackage[margin=1in]{geometry}
\usepackage{amsmath, amssymb, amsthm, esint}
\usepackage{fancyhdr}
\usepackage{tikz, tikz-3dplot}
% \usepackage{hyperref}
\usepackage{enumitem}
\usepackage{float}
\usepackage{cancel}

\setlength{\headheight}{14pt}
\fancyhf{}
\lhead{Discrete Mathematics}
\cfoot{\thepage}

\begin{document}
\pagestyle{plain}
\begin{center}
    \tableofcontents
\end{center}
\newpage
\setcounter{page}{1}
\pagestyle{fancy}
\section{Logic and Proofs}
\subsection{Propositional Logic}
Proposition is a statement that is \textbf{either} true or false, but not both at the same time. We usually represent it with variables like $p$, $q$, and $r$.\\
e.g. "The sky is blue." is a proposition, but "Listen to me" is not. 
\subsubsection*{Logical Connectives}
\begin{itemize}
    \item Negation: $\lnot \,p$. It is not the case that $p$.
    \item Conjunction: $p \land q$. "and"
    \item Disjunction: $p \lor q$. "or"
    \item Implication: $p \rightarrow q$. If $p$ then $q$, $q$ if $p$, $q$ is a consequence of $p$, $p$ only if $q$
    \item biconditional: $p \leftrightarrow q$. $(p\rightarrow q)\land(q \rightarrow p)$, $p$ if and only if $q$
\end{itemize}
\subsubsection*{Truth Table}
\begin{table}[H]
    \centering
    \hfill
    \begin{minipage}{.32\textwidth}
        \centering
        \begin{tabular}{c|c|c}
            $p$ & $q$ & $p \lor q$\\
            \hline
            T & T & T\\
            T & F & T\\
            F & T & T\\
            F & F & F
        \end{tabular}
    \end{minipage}
    \hfill
    \begin{minipage}{.32\textwidth}
        \centering
        \begin{tabular}{c|c|c}
            $p$ & $q$ & $p \land q$\\
            \hline
            T & T & T\\
            T & F & F\\
            F & T & F\\
            F & F & F
        \end{tabular}
    \end{minipage}
    \hfill
    \begin{minipage}{.32\textwidth}
        \centering
        \begin{tabular}{c|c|c}
            $p$ & $q$ & $p \rightarrow q$\\
            \hline
            T & T & T\\
            T & F & F\\
            F & T & T\\
            F & F & T
        \end{tabular}
    \end{minipage}
    \hfill
\end{table}
\subsubsection*{Example}
Find the truth value of $\boldsymbol{(p \lor q)\rightarrow \lnot \,r}$
\begin{table}[H]
    \centering
    \begin{tabular}{c|c|c|c|c|c}
        $p$ & $q$ & $r$ & $p \lor q$ & $\lnot \,r$& $(p \lor q)\rightarrow \lnot\, r$ \\
        \hline
        T & T & T & T & F & F\\
        T & T & F & T & T & T\\
        T & F & T & T & F & F\\
        T & F & F & T & T & T\\
        F & T & T & T & F & F\\
        F & T & F & T & T & T\\
        F & F & T & F & F & T\\
        F & F & F & F & T & T
    \end{tabular}
\end{table}
\subsubsection*{Logical Equivalence (L.E.)}
Two statements are logically equivalent if they always have the same truth value in every possible scenario.\\
e.g. $p$ and $q$ are biconditional, i.e. $p \leftrightarrow q$, means that $p$ and $q$ are logically equivalent. \\
There are two ways to prove L.E., one is by truth table, the other is by applying logical laws.\\
\subsubsection*{Show that $p\rightarrow q$ is logically equivalent to $\lnot \,q \rightarrow \lnot \, p$}
\begin{table}[H]
    \centering
    \begin{tabular}{c|c|c|c}
        $p$ & $q$ & $p \rightarrow q$ & $\lnot \, q\rightarrow \lnot\, p$ \\
        \hline
        T & T & T & T\\
        T & F & F & F\\
        F & T & T & T\\
        F & F & T & T\\
    \end{tabular}
\end{table}
\subsection{Application of Propositional Logic}

\end{document}
