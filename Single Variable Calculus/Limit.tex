\documentclass[11pt]{article}
\usepackage[margin=1in]{geometry}
\usepackage{amsmath, amssymb, amsthm, esint}
\usepackage{fancyhdr}
\usepackage{tikz, tikz-3dplot}
% \usepackage{hyperref}
\usepackage{enumitem}
\usepackage{float}
\usepackage{cancel}

% Page setup
\pagestyle{fancy}
\setlength{\headheight}{14pt}
\fancyhf{}
\lhead{Single Variable Calculus: Limit}
\cfoot{\thepage}

\begin{document}
\pagestyle{plain}
\begin{center}
  \tableofcontents
\end{center}
\newpage
\setcounter{page}{1}
\pagestyle{fancy}
\section{Limit of a Function}
\subsection{Definition}
Let $f$ be a function defined on an open interval containing $a$, except possibly at $a$ itself.
We say that $L$ is the \textbf{limit} of $f(x)$ as $x \to a$, and write
\[
    \lim_{x\to a}f(x) = L\] if for every $\varepsilon > 0$, there exists a $\delta > 0$ such that
\[
  0 < |x - a| < \delta \quad \Rightarrow \quad |f(x) - L| < \varepsilon.
\]
\subsection{Property}
Let $\displaystyle\lim_{x \to a} f(x) = L$ and $\displaystyle\lim_{x \to a} g(x) = M$, and let $c$ be a constant. Then the following limit properties hold:
\begin{enumerate}
    \item $ 
        \displaystyle
        \lim_{x \to a} [f(x) + g(x)] = L + M
    $    
    \item $
        \displaystyle
        \lim_{x \to a} [f(x) - g(x)] = L - M
    $
    
    \item $
        \displaystyle
        \lim_{x \to a} [c \cdot f(x)] = cL
    $
    
    \item $
        \displaystyle
        \lim_{x \to a} [f(x) \cdot g(x)] = L \cdot M
    $
    
    \item $
        \displaystyle
        \lim_{x \to a} \frac{f(x)}{g(x)} = \frac{L}{M}
    $, \text{if $M \neq 0$}
    
    \item $
        \displaystyle
        \lim_{x \to a} [f(x)]^n = L^n \quad \text{for any } n \in \mathbb{N}
    $
    
    \item $
        \displaystyle
        \lim_{x \to a} \sqrt[n]{f(x)} = \sqrt[n]{L} \quad \text{if } L \ge 0 \text{ for even } n
    $
\end{enumerate}
\subsection{One-sided Limit and Existence of a Limit}
Let $f(x)$ be a function defined near $x = a$.\\[.5em]
\textbf{Left-hand limit:}  
$
    \displaystyle
    \lim_{x \to a^-}f(x) = L
$\\
if for every $\varepsilon > 0$, there exists a $\delta > 0$ such that
\[
    0<a-x<\delta\quad\Rightarrow\quad|f(x)-L|<\varepsilon.
\]
\noindent
\textbf{Right-hand limit:}
$
    \displaystyle
    \lim_{x \to a^+} f(x) = L
$\\
if for every $\varepsilon > 0$, there exists a $\delta > 0$ such that
\[
    0<x-a<\delta\quad\Rightarrow\quad|f(x)-L|<\varepsilon.
\]
\textbf{Existence of Limit}\\
The limit of a function $f(x)$ as $x$ approaches $a$ exists if and only if the left-hand and right-hand limits exist and are equal:
\[
    \lim_{x \to a} f(x) \text{ exists } \iff \lim_{x \to a^-} f(x) = \lim_{x \to a^+} f(x)
\]

\section{Limit at Infinities}
\subsection{Infinite Limits}
If $f$ is a function defined at every number in some open inverval containing $a$
, except possibly at $a$ itself, then
\begin{itemize}
    \item $\displaystyle\lim_{x \to a} f(x) = \infty$ 
        means that $f(x)$ increases without bound as $x$
        approaches $a$.
    \item $\displaystyle\lim_{x \to a} f(x) = -\infty$ 
        means that $f(x)$ increases without bound as $x$
        approaches $a$.
    
\end{itemize}
\textbf{Limit Laws}
\begin{enumerate}
    \item If $n$ is a positive integer, then
        \begin{enumerate}
            \item $
                \displaystyle 
                \lim_{x\to0^+}\frac{1}{x^n} = \infty
            $
            \item $
                \displaystyle 
                \lim_{x\to0^-}\frac{1}{x^n} =
                \begin{cases}
                    \infty &\text{if }n\text{ is even}\\
                    -\infty &\text{if }n\text{ is odd}
                \end{cases}
            $
        \end{enumerate}
    \item if the $
           \displaystyle
            \lim_{x\to a} f(x) =c, c>0, \text{and} \lim_{x\to a} g(x) =0
            \text{, then}
        $\\
        $
            \displaystyle\lim_{x\to a}\frac{f(x)}{g(x)}=
            \begin{cases}
                \infty &\text{if }g(x)\text{ approaches 0 through positive values}\\
                -\infty &\text{if }g(x)\text{ approaches 0 through negative values}
            \end{cases}
        $
    \item if the $
            \displaystyle
            \lim_{x\to a} f(x) =c, c<0, \text{and} \lim_{x\to a} g(x) =0
            \text{, then}
        $\\
        $
            \displaystyle\lim_{x\to a}\frac{f(x)}{g(x)}=
            \begin{cases}
                -\infty &\text{if }g(x)\text{ approaches 0 through positive values}\\
                \infty &\text{if }g(x)\text{ approaches 0 through negative values}
            \end{cases}
        $
\end{enumerate}
\subsection{Limit as $x\to\infty$}
\subsubsection*{Limit at Infinity $\boldsymbol{(x\to\infty)}$}
\begin{itemize}
    \item If $f$ is a function defined at every number in some open inverval $(a,\infty)$,
        the $\displaystyle\lim_{x\to\infty}f(x)=L$ means that $L$ is the limit of $f(x)$ 
        as $x$ increases without bound.
    \item If $f$ is a function defined at every number in some open inverval $(-\infty, a)$,
        the $\displaystyle\lim_{x\to-\infty}f(x)=L$ means that $L$ is the limit of $f(x)$ 
        as $x$ decreases without bound.
\end{itemize}
\textbf{Limit Laws}\\
If $n$ is a positive integer, then
\begin{enumerate}[label=(\alph*)]
    \item $
        \displaystyle 
        \lim_{x\to\infty}\frac{1}{x^n}=0
    $
    \item $
        \displaystyle 
        \lim_{x\to-\infty}\frac{1}{x^n}=0
    $
\end{enumerate}
\subsection{Vertical and Horizontal Asymptotes}
\subsection*{Vertical Asymptotes}
A function $f(x)$ has a \textbf{vertical asymptote} at $x = a$ if at least one of the following holds:
\[
    \lim_{x \to a^-} f(x) = \pm\infty \quad \text{or} \quad \lim_{x \to a^+} f(x) = \pm\infty.
\]
This means that $f(x)$ grows without bound as $x$ approaches $a$ from the left or the right.
\subsubsection*{Horizontal Asymptotes}
A function $f(x)$ has a \textbf{horizontal asymptote} at $y = L$ if:
\[
    \lim_{x \to \infty} f(x) = L \quad \text{or} \quad \lim_{x \to -\infty} f(x) = L.
\]
This means that $f(x)$ approaches the constant value $L$ as $x$ tends to positive or negative infinity.

\section{Evaluation Techniques}
\subsection{Direct Substitution}
\[
    \lim_{x\to 2}(3x^2+2)=3(2)^2+2=14
\]
\subsection{Factorization}
\[
    \lim_{x\to 3}\frac{x^2-9}{x-3}
\]
Factorizing:
\[
    \frac{x^2-9}{x-3}=\frac{(x+3)\cancel{(x-3)}}{\cancel{x-3}}=x+3
\]
Then,
\[
    \lim_{x\to 3}\frac{x^2-9}{x-3}=\lim_{x\to 3}(x+3)=6
\]
\subsection{Rationalization}
\[
    \lim_{x\to 0}\frac{\sqrt{1+x}-1}{x}
\]
Multiply by the conjugate of numerator:
\[
    \frac{\sqrt{1+x}-1}{x}\cdot\frac{\sqrt{1+x}+1}{\sqrt{1+x}+1}=
    \frac{1+x-1}{x(\sqrt{1+x}+1)}=\frac{\cancel{x}}{\cancel{x}(\sqrt{1+x}+1)}=
    \frac{1}{\sqrt{1+x}+1}
\]
Then,
\[
    \lim_{x\to 0}\frac{\sqrt{1+x}-1}{x}=\lim_{x\to 0}\frac{1}{\sqrt{1+x}+1}=\frac{1}{2}
\]
\subsection{Use graph/table of a given function}
\[
    \begin{array}{c|cccccc}
        x & 0.9 & 0.99 & 0.999 & 1.001 & 1.01 & 1.1 \\
        \hline
        f(x) & 1.9 & 1.99 & 1.999 & 2.001 & 2.01 & 2.1
    \end{array},
    \quad\text{find $\lim_{x\to 1}f(x)$}
\]
Solution:
\[
    \lim_{x\to 1}f(x)=2
\]
\subsection{Squeeze Theorem}
Let $f(x)$, $g(x)$, and $h(x)$ be functions defined on an open interval containing $a$, 
except possibly at $a$ itself. Suppose that for all $x$ in this interval (with $x \ne a$),
\[
    f(x) \le g(x) \le h(x)
\]
and that\[
    \lim_{x \to a} f(x) = \lim_{x \to a} h(x) = L
\] 
Then,\[
    \lim_{x \to a} g(x) = L
\]
\subsubsection*{For example:}
for all $x \ne 0$,
\[
    -1 \le \sin\left(\frac{1}{x}\right) \le 1
\]
Multiplying all parts by $x^2 \ge 0$, we get
\[
    -x^2 \le x^2 \sin\left(\frac{1}{x}\right) \le x^2
\]
Since
\[
    \lim_{x \to 0} (-x^2) = 0 = \lim_{x \to 0} x^2
\]
by the \textbf{Squeeze Theorem},
\[
    \lim_{x \to 0} x^2 \sin\left(\frac{1}{x}\right) = 0
\]
\subsection{Small-Angle Approximation}
When $x\to 0$ in radian,
\[
\sin x \approx \tan x \approx x, \quad \cos x \approx 1 - \frac{x^2}{2} \Rightarrow 1 - \cos x \approx \frac{x^2}{2}
\]
\subsubsection*{For example:}
\[
    \lim_{x \to 0} \frac{1 - \cos x}{x^2}
\]
Use the identity:
\[
    \cos x = 1-2\sin^2\left(\frac{x}{2}\right)
\]
So:
\[
    \frac{1 - \cos x}{x^2} = \frac{2\sin^2\left(\frac{x}{2}\right)}{x^2}
    =\frac{2\left(\sin\left(\frac{x}{2}\right)\right)^2}{x^2}
\]
Apply small-angle approximation:
\[  
    \frac{2\left(\sin\left(\frac{x}{2}\right)\right)^2}{x^2}
    \approx \frac{2\left(\frac{x}{2}\right)^2}{x^2}
    = \frac{2}{x^2}\cdot \frac{x^2}{4}
    = \frac{2}{\cancel{x^2}} \cdot \frac{\cancel{x^2}}{4}
    = \frac{1}{2}
\]
Thus,
\[
    \lim_{x \to 0} \frac{1 - \cos x}{x^2} = \frac{1}{2}
\]
\subsection{L'Hôpital's Rule}
Suppose $\displaystyle\lim_{x \to a} f(x) = \lim_{x \to a} g(x) = 0$ or $\pm \infty$, and that
\begin{itemize}
    \item $f$ and $g$ are differentiable near $a$,
    \item $g'(x) \neq 0$ near $a$,
    \item $\displaystyle \lim_{x \to a} \frac{f'(x)}{g'(x)}$ exists.
\end{itemize}
Then,
\[
    \lim_{x \to a} \frac{f(x)}{g(x)} = \lim_{x \to a} \frac{f'(x)}{g'(x)}.
\]
\subsubsection*{For example:}
\[
    \lim_{x \to 0} \frac{e^x - 1 - x}{x^2}
\]
Apply L'Hôpital's Rule since it's $\displaystyle\frac{0}{0}$:
\[
    = \lim_{x \to 0} \frac{e^x - 1}{2x}
\]
Still $\displaystyle\frac{0}{0}$, apply L'Hôpital's Rule again:
\[
    = \lim_{x \to 0} \frac{e^x}{2} = \frac{1}{2}
\]

\section{Famous Limits}
\begin{enumerate}
    \item $
        \displaystyle
        \lim_{x\to 0}\frac{\sin(x)}{x}=1
    $
    \item $
        \displaystyle
        \lim_{x\to 0}\frac{1-\cos(x)}{x^2}=\frac{1}{2}
    $
    \item $
        \displaystyle
        \lim_{x\to 0}\frac{e^x-1}{x}=1
    $
    \item Euler's Number:\\[10pt]$
        \displaystyle
        e=\lim_{x\to\infty}\left(1+\frac{1}{x}\right)^x
    $
\end{enumerate}

\section{Continuity of a Function}
\subsection{Continuous at a Point}
A function $f$ is said to be continuous at a number $a$
if the following conditions are met:
\begin{itemize}
    \item $f(a)$ exists
    \item $\displaystyle\lim_{x\to a}f(x)$ exists
    \item $\displaystyle f(a) = \lim_{x\to a}f(x)$
\end{itemize}
\subsection{Continuous Over a Interval}
A function is continuous over an interval if it is continuous at every point in the interval.\\[.5em]
\textbf{Theorems on Continuity}
\begin{enumerate}
    \item If the function $f$ and $g$ are continuous at $a$, 
        then the functions $f+g$, $f-g$, $f\cdot g$, and $f/g$, ($g \neq 0$) are also continuous at $a$.
    \item A polynomial function is continuous everywhere.
    \item A rational function is continuous everywhere except at points where the denominator is $0$.
    \item \textbf{Intermediate Value Theorem:}\addcontentsline{toc}{subsubsection}{5.2.1\quad Intermediate Value Theorem}
        Let $f$ be a function that is continuous on the closed interval $[a, b]$. Suppose $N$ is a number such that:
        \[
            f(a) < N < f(b) \quad \text{or} \quad f(b) < N < f(a).
        \]
        Then, there exists at least one $c \in (a, b)$ such that:
        \[
            f(c) = N.
        \]
\end{enumerate}
\subsection{Types of Discontinuity}
A function $f(x)$ is said to be \textbf{discontinuous} at a point $x = a$ if the limit $\displaystyle\lim_{x \to a} f(x)$ does not exist or does not equal $f(a)$. Discontinuities can be classified into several types:
\begin{enumerate}
  \item \textbf{Removable Discontinuity:}  
  The limit $\displaystyle\lim_{x \to a} f(x)$ exists and is finite, but either $f(a)$ is not defined, or $\displaystyle f(a) \ne \lim_{x \to a} f(x)$.  
  \[
  f(x) = 
  \begin{cases}
    \displaystyle\frac{x^2 - 1}{x - 1}, & x \ne 1 \\
    0, & x = 1
  \end{cases}
  \]
  Here, $\displaystyle\lim_{x \to 1} f(x) = 2$, but $f(1) = 0 \ne 2$.

  \item \textbf{Jump Discontinuity:}  
  The left-hand limit $\displaystyle\lim_{x \to a^-} f(x)$ and right-hand limit $\displaystyle\lim_{x \to a^+} f(x)$ both exist but are not equal.  
  \[
    f(x) = 
    \begin{cases}
        1, & x < 0 \\
        2, & x \geq 0
    \end{cases}
  \]
  Then $\displaystyle\lim_{x \to 0^-} f(x) = 1$ and $\displaystyle\lim_{x \to 0^+} f(x) = 2$.

  \item \textbf{Infinite Discontinuity:}  
  The limit $\displaystyle\lim_{x \to a} f(x)$ diverges to infinity or negative infinity. That is, $f(x)$ increases or decreases without bound near $x = a$.  
  \[
    f(x) = \frac{1}{x}
    \quad \text{has an infinite discontinuity at } x = 0.
  \]

  \item \textbf{Oscillatory Discontinuity:}  
  The function oscillates infinitely near $x = a$, so the limit does not exist due to wild fluctuations.  
  \[
    f(x) = \sin\left(\frac{1}{x}\right)\text{ has an oscillatory discontinuity at } x = 0.
  \]
\end{enumerate}

\end{document}