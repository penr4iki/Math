\documentclass[11pt]{article}
\usepackage[margin=1in]{geometry}
\usepackage{amsmath, amssymb, amsthm, esint}
\usepackage{fancyhdr}
\usepackage{tikz, tikz-3dplot}
% \usepackage{hyperref}
\usepackage{enumitem}
\usepackage{float}

% Page setup
\pagestyle{fancy}
\setlength{\headheight}{14pt}
\fancyhf{}
\lhead{Single Variable Calculus: Proofs}
\cfoot{\thepage}

\newtheorem*{theorem*}{Theorem}
\newtheorem*{proof*}{Proof}

\begin{document}
\pagestyle{plain}
\begin{center}
  \tableofcontents
\end{center}
\newpage
\setcounter{page}{1}
\pagestyle{fancy}
\section*{L'Hôpital's Rule}
\addcontentsline{toc}{section}{L'Hôpital's Rule}
\subsection*{Statement of the Rule}
Let $f$ and $g$ be functions that are differentiable on an open interval $I$ containing $a$, except possibly at $a$ itself. Suppose:
\begin{itemize}
    \item $\displaystyle \lim_{x \to a} f(x) = \lim_{x \to a} g(x) = 0$ \quad \textbf{or} \quad $\displaystyle \lim_{x \to a} f(x) = \lim_{x \to a} g(x) = \pm \infty$,
    \item $g'(x) \ne 0$ for $x$ near $a$ but not equal to $a$,
    \item and the limit $\displaystyle \lim_{x \to a} \frac{f'(x)}{g'(x)}$ exists or is $\pm \infty$,
\end{itemize}
Then,
\[
    \lim_{x \to a} \frac{f(x)}{g(x)} = \lim_{x \to a} \frac{f'(x)}{g'(x)}
\]
provided the limit on the right exists.
\subsection*{Proof of L'Hôpital's Rule}
\subsubsection*{Case 1: $f(a)=g(a)=0$}
Since $g'(x)$ is non-zero near $x=a$, there is an interval $(a,b)$ such that $g'(x)$ is positive or negative for $x\in (a,b)$. Then by Cauchy's Mean Value Theorem for the interval $[a,x]$ there exists $c\in (a,b)$ such that 
\[
  \frac{f'(c)}{g'(c)}=\frac{f(x)-f(a)}{g(x)-g(a)}
\]
Since $f(a)=g(a)=0$, this reduces to 
\[
  \frac{f'(c)}{g'(c)}=\frac{f(x)}{g(x)}
\]
Now, as $x \to a^+$, we also have $c \to a^+$. If the limit $\displaystyle \lim_{x \to a^+} \frac{f'(x)}{g'(x)} = L$ exists, then by continuity of the limit,
\[
  \lim_{x \to a^+} \frac{f(x)}{g(x)} = \lim_{c \to a^+} \frac{f'(c)}{g'(c)} = L.
\]
Similarly, if we consider $x \to a^-$, then $c \to a^-$ and
\[
  \lim_{x \to a^-} \frac{f(x)}{g(x)} = \lim_{c \to a^-} \frac{f'(c)}{g'(c)} = L.
\]
If both one-sided limits exist and are equal, then the two-sided limit exists and
\[
  \lim_{x \to a} \frac{f(x)}{g(x)} = L = \lim_{x \to a} \frac{f'(x)}{g'(x)}.
\]
This proves L'Hôpital's Rule in the case where $f(a) = g(a) = 0$.
\subsubsection*{Case 2: $f(a)=g(a)=\pm\infty$}
We assume that 
\[
  \lim_{x\to a^+}\frac{f'(x)}{g'(x)}=L
\]
By the formal definition of right-hand limits, for every $\epsilon > 0$, there exists $\delta >0$ such that 
\[
  L-\epsilon < \frac{f'(x)}{g'(x)} < L +\epsilon ,\quad\forall a<x<a+\delta
\]
\subsubsection*{Case 3: $x\to\pm\infty$}
\section*{}



\end{document}