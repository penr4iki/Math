\documentclass[11pt]{article}
\usepackage[margin=1in]{geometry}
\usepackage{amsmath, amssymb, amsthm, esint, physics}
\usepackage{mathtools}
\usepackage{fancyhdr}
\usepackage{tikz, tikz-3dplot}
% \usepackage{hyperref}
\usepackage{enumitem}
\usepackage{caption}
\usepackage{float}
\usepackage{cancel}

\pagestyle{fancy}
\setlength{\headheight}{14pt}
\fancyhf{}
\lhead{Single Variable Calculus: Series}
\cfoot{\thepage}

\begin{document}
\pagestyle{plain}
\begin{center}
  \tableofcontents
\end{center}
\newpage
\setcounter{page}{1}
\pagestyle{fancy}

\section{Sequences}
\subsection{Definition of Sequences}
A sequence is an ordered list of numbers written in the form 
\[
  a_1, a_2, a_3, \dots, a_n, \quad\textbf{or}\quad a_n\quad\textbf{or}\quad\{a_n\}_{n=1}^\infty
\]
It is usually represented as a function whose domain is the set of positive integers:
\[
    a_n = f(n)
\]
\subsection{$a_n$ as $n\to\infty$}
The \textbf{limit of a sequence} as $n \to \infty$ describes the long-term behavior of the sequence:
\[
    \lim_{n \to \infty} a_n = L
\]
means that the terms of the sequence get arbitrarily close to $L$ as $n$ becomes large. If such a number $L$ exists, we say the sequence \textbf{converges} to $L$. Otherwise, it \textbf{diverges}.
\subsection{Convergence/Divergence}
A sequence $\{a_n\}$ \textbf{converges} to $L \in \mathbb{R}$ if
\[
  \lim_{n \to \infty} a_n = L \quad \Longleftrightarrow \quad \forall \varepsilon > 0, \; \exists N \in \mathbb{N}, \; \forall n > N, \; |a_n - L| < \varepsilon.
\]
Otherwise, the sequence \textbf{diverges}.
\section{Series}
\subsection{Notation}
A series is the sum of the terms of a sequence $\{a_n\}$.
Formally, the $n$-th partial sum of the series is
\[
    S_n = \sum_{i=1}^n a_i = a_1 + a_2 + \cdots + a_n
\]
If the sequence $\{S_n\}$ converges to a finite limit $S$ as $n \to \infty$, then we write
\[
    \sum_{i=1}^{\infty} a_i = L
\]
and say that the series converges to $L$. Otherwise, it diverges.
\subsection{Partial Sum}
A partial sum of a series given by 
\[
  s_n = a_1+a_2+a_3+\dots+a_n = \sum_{i=1}^{n}a_i
\]
\subsection{Types of Series}
\subsubsection*{p-Series}
The p-series is a series of the form 
\[
  1+\frac{1}{2^p}+\frac{1}{3^p}+\dots=\sum_{n=1}^\infty \frac{1}{n^p}
\]
The series converges when $p > 1$, and diverges when $p < 1$
\subsubsection*{Harmonic Series}
The harmonic series is a p-series with p = 1
\[
  1+\frac{1}{2}+\frac{1}{3}+\dots=\sum_{n=1}^\infty \frac{1}{n}
\]
It diverges
\subsubsection*{Geometric Series}
A geomeric series is a series of a form 
\[
  \sum_{n=1}^\infty ar^{n-1}
\]
The series converges when $\abs{r} < 1$.
The sum of the first $n$ terms of the series is 
\[
  s_n = \frac{a(1-r^n)}{1-r}
\]
The sum of the series is
\[
  \lim_{n\to\infty}s_n = \lim_{n\to\infty}\frac{a(1-r^n)}{1-r} = \frac{a}{1-r}
\]
\subsubsection*{Decimal Expansion}
The rational number equal to the repeating decimal is the sum of the geometric series that represents the repeating decimal. 
\[
  \begin{split}
    3.8\overline{76}&=3.8+.76+.0076+\dots\\
    &=\frac{38}{10}+\frac{76}{10^3}+\frac{76}{10^5}+\dots\\
    &=\frac{38}{10}+\sum_{n=1}^\infty\frac{76}{10^{2n+1}}\\
    &=\frac{38}{10}+\frac{76}{10^3}\cdot\frac{1}{1-\frac{1}{10^2}}\\
    &=\frac{1919}{495}
  \end{split}
\]
\subsubsection*{Telescoping}
A telescoping series is a series in which most terms cancel out when expanded, leaving only a few terms that determine the sum.  
Suppose we have a series of the form
\[
    \sum_{n=1}^{\infty} \big( a_n - a_{n+1} \big)
\]
If the sequence $\{a_n\}$ converges to a limit $L$ as $n \to \infty$, then the partial sum becomes
\[
    S_N = (a_1 - a_2) + (a_2 - a_3) + \cdots + (a_N - a_{N+1}) = a_1 - a_{N+1}
\]
Taking the limit as $N \to \infty$, we find
\[
    \sum_{n=1}^{\infty} (a_n - a_{n+1}) = a_1 - L
\]
\textbf{Example:}
Consider:
\[
    \sum_{n=1}^{\infty} \left( \frac{1}{n} - \frac{1}{n+1} \right)
\]
The partial sum is
\[
    S_N = \left(1 - \tfrac{1}{2}\right) + \left(\tfrac{1}{2} - \tfrac{1}{3}\right) + \cdots + \left(\tfrac{1}{N} - \tfrac{1}{N+1}\right)
\]
All intermediate terms cancel, leaving
\[
    S_N = 1 - \frac{1}{N+1}
\]
Taking the limit as $N \to \infty$,
\[
    \sum_{n=1}^{\infty} \left( \frac{1}{n} - \frac{1}{n+1} \right) = 1
\]
\section{Convergence Tests}
The necessary condition for a series $\{a_n\}$ to converge is that $\displaystyle\lim_{n\to\infty}a_n=0$.
\subsection{The Informal Principle}
\[
  \sum\frac{4n^3-n+1}{n^5+7n^2-6}\approx\sum\frac{4\cancel{n^3}}{\cancel{n^5}}\approx 4\sum\frac{1}{n^2}
\]
\subsection{Divergence Test}
If $\displaystyle\lim_{n\to\infty}a_n\neq0$, Then the series \textbf{diverges}.
\subsection{Integral Test}
If $a_n=f(n)$ where $f$ is continuous, decreasing, and positive on $(c,\infty]$, then $\displaystyle\sum_{n=1}^\infty$ converges $\displaystyle \iff\int_c^\infty f(x)\,dx$ exists
\subsubsection*{Example:}
\[
  \text{Let }f(n)=\frac{1}{n^2}\sin(\frac{\pi}{n})
\]
$f(n)$ is continuous, decreasing, positive on $[2,\infty)$. Then by Integral Test:
\[
  \int_{2}^{\infty}\frac{1}{x^2}\sin(\frac{\pi}{x})\,dx=\frac{1}{\pi}
\]
Thus,
\[
  \sum_{n=1}^\infty \frac{1}{n^2}\sin(\frac{\pi}{n})\textbf{ converges}
\]
\subsection{Comparison Test}
If $0\le a_n \le b_n$ and $\sum b_n$ converges, then $\sum a_n$ converges.
\subsubsection*{Example:}
\[
  \sum_{n=1}^\infty \frac{1}{n^2+5} \approx \sum_{n=1}^\infty \frac{1}{n^2}
\]
and $\displaystyle\sum_{n=1}^\infty \frac{1}{n^2}$ converges. Thus
\[
  \sum_{n=1}^\infty \frac{1}{n^2+5}\textbf{ converges}
\]
\subsection{Limit Comparison Test}
If $0<a_n$ and $0<b_n$, if $\displaystyle\lim_{n\to\infty}\frac{a_n}{b_n}=L$, then either both \textbf{converges} or \textbf{diverges}.
\subsubsection*{Example:}
\[
  \sum_{n=1}^\infty \frac{1}{4n+3}
\]
By approximating
\[
  \displaystyle\sum_{n=1}^\infty \frac{1}{4n+3} \approx \sum_{n=1}^\infty \frac{1}{n}
\]
We choose $\displaystyle\frac{1}{n}$ as $b_n$
\[
  \lim_{n\to\infty} \frac{\frac{1}{4n+3}}{\frac{1}{n}}=\lim_{n\to\infty}\frac{\cancel{n}}{4\cancel{n+3}}=\frac{1}{4}
\]
Since $\displaystyle\sum\frac{1}{n}$ diverges,
\[
  \sum_{n=1}^\infty \frac{1}{4n+3}\textbf{ diverges}
\]
\subsection{Ratio Test}
Given $a_n$ and $a_{n+1}$, we find the limit of their absolute ratio, i.e. $\displaystyle\lim_{n\to\infty}\left|\frac{a_{n+1}}{a_n}\right|$. 
\[
  \lim_{n\to\infty}\left|\frac{a_{n+1}}{a_n}\right|=
  \begin{cases}
    <1, &\textbf{Converges}\\
    =1, &\textbf{Inconclusive}\\
    >1, &\textbf{Diverges}
  \end{cases}
\]
\subsubsection*{Example:}
\[
  \sum_{n=1}^\infty \frac{2^n(n+1)}{n!}
\]
We find the limit of their absolute ratio
\[
  \begin{aligned}
    &\lim_{n\to\infty} \left(\frac{2^{n+1}(n+2)}{(n+1)!}\cdot\frac{n!}{2^n(n+1)}\right)\\
    =&\lim_{n\to\infty} \left(
        \frac{2^{\cancel{n}+1}(n+2)}{(n+1)\cancel{!}}\cdot\frac{\cancel{n!}}{\cancel{2^n}\cancel{(n+1)}}
      \right)\\
    =&\lim_{n\to\infty}\frac{2(n+2)}{(n+1)^2}\\
    =&\,0<1
  \end{aligned}
\]
Thus,
\[
  \sum_{n=1}^\infty \frac{2^n(n+1)}{n!}\textbf{ converges}
\]
\subsection{Root Test}
Given $\displaystyle\sum a_n$ , we find the limit of the n-th root of $a_n$, i.e. $\displaystyle\lim_{n\to\infty}\sqrt[n]{\left|{a_n}\right|}$.
The limit measures the asymptotic size of the terms by looking at their n-th root.
\[
  \lim_{n\to\infty}\sqrt[n]{\left|{a_n}\right|}=
  \begin{cases}
    <1, &\textbf{Converges}\\
    =1, &\textbf{Inconclusive}\\
    >1, &\textbf{Diverges}
  \end{cases}
\]
\subsubsection*{Example:}
\[
  \sum_{n=1}^{\infty} \frac{n^3}{5^n} \left(1+\frac{1}{n}\right)^{n}.
\]
Apply Root Test:
\[
  \begin{aligned}
    &\lim_{n\to\infty}\sqrt[n]{|a_n|}\\
    =&\lim_{n\to\infty}\sqrt[n]{n^3} \cdot \sqrt[n]{\frac{1}{5^n}} \cdot \sqrt[n]{\left(1+\frac{1}{n}\right)^{n}}
  \end{aligned}
\]
Evaluate:
\[
  \begin{aligned}
    &\sqrt[n]{n^3} \to 1\text{ because } n^{3/n} \to 1.\\[10pt]
    &\sqrt[n]{\frac{1}{5^n}} = \frac{1}{5}.\\[10pt]
    &\sqrt[n]{\left(1+\frac{1}{n}\right)^{n}} \to e^{1/n} \to 1\\[10pt]
    &\Rightarrow\lim_{n\to\infty}\sqrt[n]{|a_n|}=1 \cdot \left(\frac{1}{5}\right) \cdot 1 = \frac{1}{5}<1
  \end{aligned}
\]
Thus,
\[
  \sum_{n=1}^\infty \frac{2^n(n+1)}{n!}=\textbf{ converges}
\]
\subsection{Alternating Series Test}
\subsection{Absolute vs Conditional Convergence}
\section{Power Series}
\subsection{Definition}
\subsection{Radius and Interval of Convergence}
\subsection{Differentiation and Integration}
\section{Taylor and Maclaurin Series}
\subsection{Taylor Series}
A Taylor series is an infinite sum that represents a function as a power series centered at a point $a$. If a function $f(x)$ is infinitely differentiable at $x = a$, then its Taylor series is given by:
\[
  f(x)=\sum_{n=0}^\infty\frac{f^n(a)}{n!}(x-a)^n
\]
This expansion approximates the function near $x=a$
\subsection{Maclaurin Series}
A Maclaurin Series is a special case of Taylor Series centered at $x=0$
\[
  f(x) = \sum_{n=0}^{\infty} \frac{f^{(n)}(0)}{n!} x^n
\]
\subsection{Common Maclaurin Series}
\begin{itemize}
  \item$\displaystyle e^x = \sum_{n=0}^\infty \frac{x^n}{n!}$
  \item$\displaystyle \sin(x) = \sum_{n=0}^\infty \frac{(-1)^n}{(2n+1)!}\,x^{2n+1}$
  \item$\displaystyle \cos(x) = \sum_{n=0}^\infty \frac{(-1)^n}{2n!}\,x^{2n}$
  \item$\displaystyle \frac{1}{1-x} = \sum_{n=0}^\infty x^n$
  \item$\displaystyle \frac{1}{1+x} = \sum_{n=0}^\infty (-1)^nx^n$
  \item$\displaystyle \ln(1-x) = \sum_{n=0}^\infty \frac{1}{n+1}x^{n+1}$
  \item$\displaystyle \ln(1+x) = \sum_{n=0}^\infty \frac{(-1)^n}{n+1}x^{n+1}$
\end{itemize}
\subsection{Lagrange Error Bound for Taylor Series}
The Lagrange error bound provides a way to estimate how close the Taylor polynomial $T_n(x)$ is to the actual function $f(x)$.  
Let $f$ be a function with $(n+1)$ continuous derivatives on an interval containing $a$ and $x$. The Taylor polynomial of degree $n$ centered at $a$ is
\[
T_n(x) = f(a) + f'(a)(x-a) + \frac{f''(a)}{2!}(x-a)^2 + \cdots + \frac{f^{(n)}(a)}{n!}(x-a)^n.
\]
The remainder/error term in Lagrange form is
\[
  R_n(x) = \frac{f^{(n+1)}(c)}{(n+1)!}(x-a)^{n+1}
\]
for some $c\in[x,a]$.
\subsubsection*{Error Bound}
\[
  |R_n(x)| \le \frac{\text{M}}{(n+1)!} |x-a|^{n+1}\text{, where }M\text{ is max(}|f^{n+1}(c)|\text{)}
\]for some $c\in[x,a]$
\subsection{Limits and Approximations}
\section{Applications}
\subsection{Numerical Approximation}
\subsection{Solving ODEs}
\subsection{Non-elementary Integrals}

\end{document}