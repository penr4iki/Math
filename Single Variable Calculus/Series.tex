\documentclass[11pt]{article}
\usepackage[margin=1in]{geometry}
\usepackage{amsmath, amssymb, amsthm, esint, physics}
\usepackage{fancyhdr}
\usepackage{tikz, tikz-3dplot}
% \usepackage{hyperref}
\usepackage{enumitem}
\usepackage{caption}
\usepackage{float}

% Page setup
\pagestyle{fancy}
\setlength{\headheight}{14pt}
\fancyhf{}
\lhead{Single Variable Calculus: Series}
\cfoot{\thepage}

\begin{document}
\pagestyle{plain}
\begin{center}
  \tableofcontents
\end{center}
\newpage
\setcounter{page}{1}
\pagestyle{fancy}

\section{Sequences}
\subsection{Definition of Sequences}
A sequence is an ordered list of numbers written in the form 
\[
  a_1, a_2, a_3, \dots, a_n, \quad\textbf{or}\quad a_n\quad\textbf{or}\quad\{a_n\}_{n=1}^\infty
\]
It is usually represented as a function whose domain is the set of positive integers:
\[
    a_n = f(n)
\]
\subsection{$a_n$ as $n\to\infty$}
The \textbf{limit of a sequence} as $n \to \infty$ describes the long-term behavior of the sequence:
\[
    \lim_{n \to \infty} a_n = L
\]
means that the terms of the sequence get arbitrarily close to $L$ as $n$ becomes large. If such a number $L$ exists, we say the sequence \textbf{converges} to $L$. Otherwise, it \textbf{diverges}.
\subsection{Convergence/Divergence}
A sequence $\{a_n\}$ \textbf{converges} to $L \in \mathbb{R}$ if
\[
  \lim_{n \to \infty} a_n = L \quad \Longleftrightarrow \quad \forall \varepsilon > 0, \; \exists N \in \mathbb{N}, \; \forall n > N, \; |a_n - L| < \varepsilon.
\]
Otherwise, the sequence \textbf{diverges}.
\subsection{Monotone and Bounded Sequences}
\section{Series}
\subsection{Notation}
\subsection{Partial Sum}
\subsection{Types of Series}
\subsection{Telescoping}
\section{Convergence Tests}
\subsection{Divergence Test}
\subsection{Comparison}
\subsection{Limit Comparison Test}
\subsection{Ratio Test}
\subsection{Root Test}
\subsection{Integral Test}
\subsection{Alternating Series Test}
\subsection{Absolute vs Conditional Convergence}
\section{Power Series}
\subsection{Definition}
\subsection{Radius and Interval of Convergence}
\subsection{Differentiation and Integration}
\section{Taylor and Maclaurin Series}
\subsection{Taylor Series}
A Taylor series is an infinite sum that represents a function as a power series centered at a point $a$. If a function $f(x)$ is infinitely differentiable at $x = a$, then its Taylor series is given by:
\[
  f(x)=\sum_{n=0}^\infty\frac{f^n(a)}{n!}(x-a)^n
\]
This expansion approximates the function near $x=a$
\subsection{Maclaurin Series}
A Maclaurin Series is a special case of Taylor Series centered at $x=0$
\[
  f(x) = \sum_{n=0}^{\infty} \frac{f^{(n)}(0)}{n!} x^n
\]
\subsection{Common Expansions}

\subsection{Lagrange Error Bound for Taylor Series}
\subsection{Limits and Approximations}
\section{Applications}
\subsection{Numerical Approximation}
\subsection{Solving ODEs}
\subsection{Non-elementary Integrals}

\end{document}