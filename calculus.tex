\documentclass[11pt]{article}
\usepackage[margin=1in]{geometry}
\usepackage{amsmath, amssymb, amsthm, esint}
\usepackage{fancyhdr}
\usepackage{tikz, tikz-3dplot}
% \usepackage{hyperref}
\usepackage{enumitem}

% Page setup
\pagestyle{fancy}
\setlength{\headheight}{14pt}
\fancyhf{}
\lhead{Calculus}
\rhead{\today}
\cfoot{\thepage}

% Theorem setup
\newtheorem{theorem}{Theorem}[section]

\begin{document}
\pagestyle{plain}
\begin{center}
  \tableofcontents
\end{center}

\newpage
\setcounter{page}{1}
\pagestyle{fancy}

\section{Limit}
\subsection{Limit of a function}
\subsubsection{Definition}
Let $f$ be a function defined on an open interval containing $a$, except possibly at $a$ itself. 
Then \[\lim_{x\to a}f(x) = L\] if for every $\varepsilon > 0$, there exists a $\delta > 0$ such that
\[
0 < |x - a| < \delta \quad \Rightarrow \quad |f(x) - L| < \varepsilon.
\]
\subsubsection{Property}
Let $\displaystyle\lim_{x \to a} f(x) = L$ and $\displaystyle\lim_{x \to a} g(x) = M$, and let $c$ be a constant. Then the following limit properties hold:
\begin{enumerate}
    \item $ 
        \displaystyle
        \lim_{x \to a} [f(x) + g(x)] = L + M
    $    
    \item $
        \displaystyle
        \lim_{x \to a} [f(x) - g(x)] = L - M
    $
    
    \item $
        \displaystyle
        \lim_{x \to a} [c \cdot f(x)] = cL
    $
    
    \item $
        \displaystyle
        \lim_{x \to a} [f(x) \cdot g(x)] = L \cdot M
    $
    
    \item $
        \displaystyle
        \lim_{x \to a} \frac{f(x)}{g(x)} = \frac{L}{M}
    $, \text{if $M \neq 0$}
    
    \item $
        \displaystyle
        \lim_{x \to a} [f(x)]^n = L^n \quad \text{for any } n \in \mathbb{N}
    $
    
    \item $
        \displaystyle
        \lim_{x \to a} \sqrt[n]{f(x)} = \sqrt[n]{L} \quad \text{if } L \ge 0 \text{ for even } n
    $
\end{enumerate}
\subsubsection{One-sided Limit and Existence of a Limit}
Let $f(x)$ be a function defined near $x = a$.\\[.5em]
\noindent
\textbf{Left-hand limit:}  
$
    \displaystyle
    \lim_{x \to a^-}f(x) = L
$\\
if for every $\varepsilon > 0$, there exists a $\delta > 0$ such that
\[
    0<a-x<\delta\quad\Rightarrow\quad|f(x)-L|<\varepsilon.
\]
\noindent
\textbf{Right-hand limit:}\\
$
    \lim_{x \to a^+} f(x) = L
$\\
if for every $\varepsilon > 0$, there exists a $\delta > 0$ such that
\[
    0<x-a<\delta\quad\Rightarrow\quad|f(x)-L|<\varepsilon.
\]
\textbf{Existence of Limit}\\
The limit of a function $f(x)$ as $x$ approaches $a$ exists if and only if the left-hand and right-hand limits exist and are equal:
\[
    \lim_{x \to a} f(x) \text{ exists } \iff \lim_{x \to a^-} f(x) = \lim_{x \to a^+} f(x)
\]
\subsubsection{Evaluating Limit}
\begin{enumerate}
    \item Substitute directly
    \item Factoring and simplifying
    \item Multiply by the conjugate of numerator or denominator
    \item Use graph/table of a given function
\end{enumerate}
\subsubsection{Squeeze Theorem}
\begin{theorem}[Squeeze Theorem]
Let $f(x)$, $g(x)$, and $h(x)$ be functions defined on an open interval containing $a$, 
except possibly at $a$ itself. Suppose that for all $x$ in this interval (with $x \ne a$),
\[f(x) \le g(x) \le h(x),\]
and that\[\lim_{x \to a} f(x) = \lim_{x \to a} h(x) = L.\] 
Then,\[\lim_{x \to a} g(x) = L.\]
\end{theorem}

\bigskip
\noindent
\textbf{For example:}\\
for all $x \ne 0$,
\[
-1 \le \sin\left(\frac{1}{x}\right) \le 1.
\]
Multiplying all parts by $x^2 \ge 0$, we get
\[-x^2 \le x^2 \sin\left(\frac{1}{x}\right) \le x^2.\]
Since
\[\lim_{x \to 0} (-x^2) = 0 = \lim_{x \to 0} x^2,\]
by the \textbf{\textit{Squeeze Theorem}},
\[\lim_{x \to 0} x^2 \sin\left(\frac{1}{x}\right) = 0.\]
\subsection{Limit with Infinities}
\subsubsection{Infinite Limits}
If $f$ is a function defined at every number in some open inverval containing $a$
, except possibly at $a$ itself, then
\begin{itemize}
    \item $\displaystyle\lim_{x \to a} f(x) = \infty$ 
        means that $f(x)$ increases without bound as $x$
        approaches $a$.
    \item $\displaystyle\lim_{x \to a} f(x) = -\infty$ 
        means that $f(x)$ increases without bound as $x$
        approaches $a$.
    
\end{itemize}
\textbf{Limit Theorems}
\begin{enumerate}
    \item If $n$ is a positive integer, then
        \begin{enumerate}
            \item $
                \displaystyle 
                \lim_{x\to0^+}\frac{1}{x^n} = \infty
            $
            \item $
                \displaystyle 
                \lim_{x\to0^-}\frac{1}{x^n} =
                \begin{cases}
                    \infty &\text{if }n\text{ is even}\\
                    -\infty &\text{if }n\text{ is odd}
                \end{cases}
            $
        \end{enumerate}
    \item if the $\displaystyle
        \lim_{x\to a} f(x) =c, c>0, \text{and} \lim_{x\to a} g(x) =0
        \text{, then}$\\$
            \displaystyle\lim_{x\to a}\frac{f(x)}{g(x)}=
        \begin{cases}
            \infty &\text{if }g(x)\text{ approaches 0 through positive values}\\
            -\infty &\text{if }g(x)\text{ approaches 0 through negative values}
        \end{cases}$
    \item if the $\displaystyle
        \lim_{x\to a} f(x) =c, c<0, \text{and} \lim_{x\to a} g(x) =0
        \text{, then}$\\$
            \displaystyle\lim_{x\to a}\frac{f(x)}{g(x)}=
        \begin{cases}
            -\infty &\text{if }g(x)\text{ approaches 0 through positive values}\\
            \infty &\text{if }g(x)\text{ approaches 0 through negative values}
        \end{cases}$
\end{enumerate}
\subsubsection{Limit at Infinities}
\textbf{Limit at Infinity $\boldsymbol{(x\to\infty)}$}
\begin{itemize}
    \item If $f$ is a function defined at every number in some open inverval $(a,\infty)$,
        the $\displaystyle\lim_{x\to\infty}f(x)=L$ means that $L$ is the limit of $f(x)$ 
        as $x$ increases without bound.
    \item If $f$ is a function defined at every number in some open inverval $(-\infty, a)$,
        the $\displaystyle\lim_{x\to-\infty}f(x)=L$ means that $L$ is the limit of $f(x)$ 
        as $x$ decreases without bound.
\end{itemize}
\textbf{Limit Theorems}\\
If $n$ is a positive integer, then
\begin{enumerate}[label=(\alph*)]
    \item $
        \displaystyle 
        \lim_{x\to\infty}\frac{1}{x^n}=0
    $
    \item $
        \displaystyle 
        \lim_{x\to-\infty}\frac{1}{x^n}=0
    $
\end{enumerate}
\subsubsection{Vertical and Horizontal Asymptotes}
\textbf{Vertical Asymptotes}\\
A function $f(x)$ has a \textbf{vertical asymptote} at $x = a$ if at least one of the following holds:
\[
    \lim_{x \to a^-} f(x) = \pm\infty \quad \text{or} \quad \lim_{x \to a^+} f(x) = \pm\infty.
\]
\noindent
This means that $f(x)$ grows without bound as $x$ approaches $a$ from the left or the right.\\[.5em]
\textbf{Horizontal Asymptotes}\\
A function $f(x)$ has a \textbf{horizontal asymptote} at $y = L$ if:
\[
    \lim_{x \to \infty} f(x) = L \quad \text{or} \quad \lim_{x \to -\infty} f(x) = L.
\]
\noindent
This means that $f(x)$ approaches the constant value $L$ as $x$ tends to positive or negative infinity.
\subsection{Continuity}

\section{Derivatives}


\section{Integrals}
\end{document}
